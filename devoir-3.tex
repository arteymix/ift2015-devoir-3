\documentclass[french]{article}
\usepackage[utf8]{inputenc}
\usepackage[a4paper]{geometry}
\usepackage[french]{babel}

\title{Devoir 3 \\Structure de Données}
\author{Guillaume Poirier-Morency p1053380 \\ Vincent Antaki p1038646}

\renewcommand{\thesubsection}{\thesection.\alph{subsection}}
\begin{document}

\maketitle
\subsection{Résumé}

Implémentation d'une file à priorités multiples et tests relatifs.

\section{}


\section{}

La classe Queue

Notre implémentation commence avec la classe Queue qui impémente une structure 
de liste chainée simple et les fonctions enqueue() dequeue() et remove(). Les 
fonctions built-in python 
(\_\_len\_\_, \_\_iter\_\_, \_\_bool\_\_, \_\_next\_\_, \_\_contains\_\_, ...) 
on été utilisée pour améliorer la rapidité d'exécution de notre programme et 
ainsi qu'en faciliter l'implémentation.

CasinoQueue

Les instances de la classe CasinoQueue sont des files à 3 priorités. Elles 
possèdent les fonctions de base des files : enqueue() et dequeue(). Elle 
possède 3 files chainées de la classe Queue correspondant à ses trois 
priorités (table brisée, changement de table et le reste). Les éléments mis 
dans les files sont des tuplets de joueur, temps d'entrée dans la file et, 
dans le cas de la file pour changement de table, la table désirée.  

Lorsque appelée, la fonction dequeue retire, en tout respect de l'énoncé, une 
personne de la file ou retourne une exception si cela est impossible. 
dequeue() possède un ordre constant lorsque il y a des éléments dans la queue 
pour table brisée et a une ordre linéaire par rapport au nombre de joueur dans 
la queue pour changement de table dans tous les autres cas.

Chaque appel de la fonction enqueue vérifie que le nom entrée n'est pas un 
doublon d'un nom existant déjà dans le casino. Si ce n'est pas le cas, les 
paramètres table et broken détermineront à quelle file sera enqueue le joueur.

Il nous aurait été possible d'implémenter la vérification des doublons par une 
itération à travers les 3 queues. Pour cause de mauvaise complexité, nous 
avons refusé cette option. Il nous aurait été possible de faire un arbre qui 
stocke les noms de tout les joueurs qui sont dans le casino et qui font des 
recherches en O(log n). Pour cause de flemmardise, nous avons refusé cette 
option. Nous avons implémenté \_\_contains\_\_ qui test l'appartenance à un 
objet à self.players (un set!)lors de l'entrée d'un nouveau joueur dans le 
casino (ajout d'un joueur à la normal\_queue).



\end{document}
